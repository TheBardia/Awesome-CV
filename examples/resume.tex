%!TEX TS-program = xelatex
%!TEX encoding = UTF-8 Unicode
% Awesome CV LaTeX Template for CV/Resume
%
% This template has been downloaded from:
% https://github.com/posquit0/Awesome-CV
%
% Author:
% Claud D. Park <posquit0.bj@gmail.com>
% http://www.posquit0.com
%
% Template license:
% CC BY-SA 4.0 (https://creativecommons.org/licenses/by-sa/4.0/)
%


%-------------------------------------------------------------------------------
% CONFIGURATIONS 
%-------------------------------------------------------------------------------
% A4 paper size by default, use 'letterpaper' for US letter
\documentclass[11pt, letterpaper]{awesome-cv}

% Configure page margins with geometry
\geometry{left=1.6cm, top=0.3cm, right=1.6cm, bottom=0.6cm, footskip=0.1cm}

% Specify the location of the included fonts
\fontdir[fonts/]

% Color for highlights
% Awesome Colors: awesome-emerald, awesome-skyblue, awesome-red, awesome-pink, awesome-orange
%                 awesome-nephritis, awesome-concrete, awesome-darknight
% \colorlet{awesome}{awesome-red}
% Uncomment if you would like to specify your own color
\definecolor{awesome}{HTML}{f08d3a}
\definecolor{secondary}{HTML}{000000}


% Colors for text
% Uncomment if you would like to specify your own color
% \definecolor{darktext}{HTML}{414141}
% \definecolor{text}{HTML}{333333}
% \definecolor{graytext}{HTML}{5D5D5D}
% \definecolor{lighttext}{HTML}{999999}

\AddToShipoutPictureBG{%
  \begin{tikzpicture}[overlay,remember picture]
  \draw[secondary, line width=24pt]
      ($ (current page.north west) + (12pt,0pt) $)
      --
      ($ (current page.south west) + (12pt,0pt) $);
  \draw[primary, line width=6pt]
      ($ (current page.north west) + (24pt,0pt) $)
      --
      ($ (current page.south west) + (24pt,0pt) $);
  \end{tikzpicture}
}

% Set false if you don't want to highlight section with awesome color
\setbool{acvSectionColorHighlight}{true}

% If you would like to change the social information separator from a pipe (|) to something else
\renewcommand{\acvHeaderSocialSep}{\quad\textbar\quad}


%-------------------------------------------------------------------------------
%	PERSONAL INFORMATION
%	Comment any of the lines below if they are not required
%-------------------------------------------------------------------------------
% Available options: circle|rectangle,edge/noedge,left/right
% \photo[rectangle,noedge,right]{./profile.png}
\name{Ali}{Niaki}
% \position{Software Architect{\enskip\cdotp\enskip}Security Expert}
% \position{Software + Security{\enskip\cdotp\enskip}Computer Security Graduate}
\address{10 Leora Court, Richmond Hill, Ontario, Canada}

\mobile{(+1) 647-964-1375}
\email{niakicom@gmail.com}
% \homepage{www.aliniaki.me}
\github{saaniaki}
\linkedin{saaniaki}
% \gitlab{gitlab-id}
% \stackoverflow{SO-id}{SO-name}
% \twitter{@twit}
% \skype{skype-id}
% \reddit{reddit-id}
% \medium{madium-id}
% \googlescholar{googlescholar-id}{name-to-display}
%% \firstname and \lastname will be used
% \googlescholar{googlescholar-id}{}
% \extrainfo{extra informations}

% \quote{``Be the change that you want to see in the world."}

% The company being applied to
\setcompanynamewiths{Okta's}

%-------------------------------------------------------------------------------
\begin{document}

% Print the header with above personal informations
% Give optional argument to change alignment(C: center, L: left, R: right)
\makecvheader[C]

% Print the footer with 3 arguments(<left>, <center>, <right>)
% Leave any of these blank if they are not needed
\makecvfooter
  {} % \thepage
  {Ali Niaki~~~·~~~Résumé}
  {\today}


%-------------------------------------------------------------------------------
%	CV/RESUME CONTENT
%	Each section is imported separately, open each file in turn to modify content
%-------------------------------------------------------------------------------
%-------------------------------------------------------------------------------
%	SECTION TITLE
%-------------------------------------------------------------------------------
% \cvsection{Profile Summary}


%-------------------------------------------------------------------------------
%	CONTENT
%-------------------------------------------------------------------------------
\begin{cvparagraph}
        \begin{summaryitems}
            \item \begin{center} \importantstyle{Computer Science} at YorkU (graduated 2020)\end{center}
            \item \begin{center} Experience with numerous coding languages such as \importantstyle{C/C++}, \importantstyle{Java}, \importantstyle{Javascript}, \importantstyle{Python}, and more in designging \importantstyle{applications}, \importantstyle{Android apps}, \importantstyle{databases}, \importantstyle{microservices}, \importantstyle{REST APIs} and more.\end{center}
            \item \begin{center} Passionate and eager to constantly \importantstyle{grow} in the field and \importantstyle{expand} my list of capabilities \end{center}
        \end{summaryitems}
\end{cvparagraph}

\vspace{-3mm}

%-------------------------------------------------------------------------------
%	SECTION TITLE
%-------------------------------------------------------------------------------
\cvsection{Skills \& Concepts}


%-------------------------------------------------------------------------------
%	CONTENT
%-------------------------------------------------------------------------------
\begin{cvskills}

%---------------------------------------------------------
\cvskill
  {Languages} % Category
  {\importantstyle{C}, C++, \importantstyle{Java}, \importantstyle{JavaScript}, \importantstyle{SQL}, \importantstyle{Python}, Typescript, Eiffel, \importantstyle{HTML}, CSS, PHP, smalltalk, drRacket, QBasic} % Skills

%---------------------------------------------------------
\cvskill
  {Concepts} % Category
  {\importantstyle{Linux}, Version-Control(\importantstyle{Git}), Concurrency, \importantstyle{Multithreading}, Synchronization, Microservices, REST APIs, Databases, \importantstyle{UX/UI}, VR, Game Engines (UE4, Unity)} % Skills

%---------------------------------------------------------
\cvskill
  {Frameworks} % Category
  {\importantstyle{Spring Boot}} % Skills

%---------------------------------------------------------
\end{cvskills}

%-------------------------------------------------------------------------------
%	SECTION TITLE
%-------------------------------------------------------------------------------
\cvsection{Education}


%-------------------------------------------------------------------------------
%	CONTENT
%-------------------------------------------------------------------------------
\begin{cventries}

%---------------------------------------------------------
  \cventry
    {York University} % Institution
    {B.Sc. Computer Science} % Degree
    {Jan. 2017 - Apr. 2019} % Date(s)
    {North York, Ontario, CA} % Location
    {
      \begin{cvitems} % Description(s) bullet points
        \item {Relevant topics include: \importantstyle{C}, \importantstyle{Java}, \importantstyle{Operating Systems}, Web Development, Databases, UX/UI, VR and more}
        \item {Developed 3 distinct \importantstyle{Deep Learning} algorithms during a self-study course: a \importantstyle{stock prediction algorithm}, an \importantstyle{animal classification} algorithm, and the final one being a self-taught \importantstyle{emotion detection} algorithm that would utilize the webcam and guess the user's facial expression}
        \item {Developed algorithm used intandem with a \importantstyle{robot arm} with 3 degrees of freedom and a webcam, and as many colored 3d printed cubes on the table. The algorithm would move the robot arm to pick the cubes one at a time and sort and stack them based on color}
      \end{cvitems}
    }


%---------------------------------------------------------
  \cventry
  {Ryerson University} % Institution
  {Computer Programming Applications Certificate} % Degree
  {Jan. 2016 - Dec. 2016} % Date(s)
  {Toronto, Ontario, CA} % Location
  {
    \begin{cvitems} % Description(s) bullet points
      \item {Notably developed an \importantstyle{Android application} that would accompany gym trainers at work, \importantstyle{keeping tracking of their users, types of workouts, schedule, and number of reps/weight} of the exercise performed by clients}
    \end{cvitems}
  }

%---------------------------------------------------------
\end{cventries}
%-------------------------------------------------------------------------------
%	SECTION TITLE
%-------------------------------------------------------------------------------
\cvsection{Work Experience}


%-------------------------------------------------------------------------------
%	CONTENT
%-------------------------------------------------------------------------------
\begin{cventries}

%---------------------------------------------------------
  \cventry
    {York University} % Organization
    {Computer Security Research Assistance} % Job title
    {Jan. 2020 - Aug. 2020} % Date(s)
    {North York, Ontario, CA} % Location
    {
      \begin{cvitems} % Description(s) of tasks/responsibilities
        \item {Malwareless Web Analytics Pollution (MWAP) Tools Analysis}
          \begin{cvsubitems}
            \item {Multiple attacking tools and headless browsers analyzed.}
            \item {Overall feasibility and effectiveness of conducting successful MWAP attacks evaluated where the attacker can generate a large number of fake requests from a single fully controlled machine.}
            \item {New \importantstyle{multithreaded MWAP attack tool developed} (\underline{\href{https://github.com/saaniaki/eloki/tree/master/eloki}{eLoki}}) in Java using \importantstyle{HtmlUnit} and \importantstyle{Selenium} to precisely assess \importantstyle{Google Analytics} and \importantstyle{AWStats} vulnerability to fake requests.}
            \item {Currently (and until the end of August) developing software module for \underline{\href{https://github.com/saaniaki/eloki/tree/master/eloki}{eLoki}} to record \importantstyle{genuine human-user generated mouse movement} and actions using Selenium WebDriver. Professor \underline{\href{http://www.cse.yorku.ca/~vlajic/}{Natalia Vlajic}} will use final software against known bot-detecting tools, including Google Analytics and \importantstyle{Google reCAPTCHA v3}.}
          \end{cvsubitems}
      \end{cvitems}
    }

%---------------------------------------------------------
  \cventry
    {Tread2Go Inc.} % Organization
    {Co-founder \& CTO} % Job title
    {Jun. 2019 - Current} % Date(s)
    {Richmond Hill, Ontario, CA} % Location
    {
      \begin{cvitems} % Description(s) of tasks/responsibilities
        \item {\importantstyle{Architected and developed mini ERP RESTful API} using Spring boot MVC, Spring Cloud and deployed on AWS Beanstalk.}
          \begin{cvsubitems}
            \item {Tread2Go monolithic backend was designed to schedule orders and \importantstyle{allocate resources to optimize operations and minimize costs.}}
            \item {Software needed to be \importantstyle{highly scalable} and convertible to \importantstyle{microservices}.}
            \item {To make the development process faster, integrated and used many third-party APIs (Firebase, Google Cloud APIs, Nexmo, Stripe and more).}
            \item {Applied best practises and design patterns while designing and developing \importantstyle{RBAC to handle authorization} using \importantstyle{Spring Security}.}
          \end{cvsubitems}
        \item {Initiated Tread2Go \importantstyle{Angular PWA} and managed Angular front-end team using Trello, GitHub and Swagger; improved interpersonal skills to lead the project and \importantstyle{soft skills} to manage the stressful environment.}
          \begin{cvsubitems}
            \item {Repetitively converted business requirements to practical development plans, \importantstyle{managed and executed development plans using the defined budget}, enabled the company to continue working with \importantstyle{lean} and \importantstyle{agile} strategy.}
          \end{cvsubitems}
        \item {Deployed and maintained PWA and server-side services, orchestrated and managed DNS, Security Policies, S3, EC2, Beanstalk, RDS and SES.}
          \begin{cvsubitems}
            \item {Applied and setup \importantstyle{DevOps} and development pipelines using Amplify and GitHub, significantly eased \importantstyle{PWA delivery}.}
          \end{cvsubitems}
      \end{cvitems}
    }

%---------------------------------------------------------
  \cventry
    {Scotiabank} % Organization
    {Information Security Analyst} % Job title
    {Jan. 2019 - Apr. 2019} % Date(s)
    {Scarborough, Ontario, CA} % Location
    {
      \begin{cvitems} % Description(s) of tasks/responsibilities
        \item {\importantstyle{Developed a robust cryptography tool} specifically for the Information Security and Control department using Java \importantstyle{without any framework} to include the least number of external libraries as a security requirement.}
          \begin{cvsubitems}
            \item {Cryptool is compatible with most common \importantstyle{OpenSSL} standards and leverages from both \importantstyle{Symmetric and Asymmetric cryptography}.}
            \item {Cryptool has different modes also to ensure \importantstyle{data integrity via digital signature}.}
          \end{cvsubitems}
        \item {Helped to analyze and test application security, including:}
          \begin{cvsubitems}
            \item {Static Application Security Testing (\importantstyle{SAST}) via Micro Focus Fortify and Blackduck.}
            \item {Dynamic Application Security Testing (\importantstyle{DAST}) via Micro Focus Fortify Webinspect.}
          \end{cvsubitems}
        \item {Integrated some new security analysis and management software by learning and utilizing Python 3; helped the security team analyze and manage static source code vulnerabilities more efficiently.}
          \begin{cvsubitems}
            \item {Utilized Jira and Bitbucket and started to learn and adapt \importantstyle{Jenkins} to make the new application delivery process more efficient and robust.}
          \end{cvsubitems}
      \end{cvitems}
    }

%---------------------------------------------------------
  \cventry
    {York University} % Organization
    {Teaching Assistance} % Job title
    {Sep. 2018 - Dec. 2018} % Date(s)
    {North York, Ontario, CA} % Location
    {
      \begin{cvitems} % Description(s) of tasks/responsibilities
        \item {Assisted in the “Programming for Mobile Computing” course by helping students learn about Java 8 and \importantstyle{Android Development}, which significantly contributed to improving \importantstyle{interpersonal skills}.}
      \end{cvitems}
    }

%---------------------------------------------------------
  \cventry
    {Cube Angle Inc.} % Organization
    {Full Stack Developer} % Job title
    {Jul. 2018 - Sep. 2018} % Date(s)
    {Markham, Ontario, CA} % Location
    {
      \begin{cvitems} % Description(s) of tasks/responsibilities
        \item {Architected and developed Trade Dimension Dashboard (TDD) using Angular and Spring framework (REST API).}
          \begin{cvsubitems}
            \item {TDD was a \importantstyle{Cloud-based web application} mainly designed to visualize trade data stored on Amazon Redshift.}
          \end{cvsubitems}
        \item {Learned and utilized cloud computing concepts and various AWS tools to deploy Trade Dimension Dashboard, including:}
          \begin{cvsubitems}
            \item {\importantstyle{AWS Cognito}, Redshift, RDS and API Gateway and \importantstyle{Ubuntu} EC2 to make backend available.}
            \item {“amplifyjs” and “awsmobile” to connect the frontend to AWS Cognito and Mobile Hub.}
          \end{cvsubitems}
      \end{cvitems}
    }

%---------------------------------------------------------
  \cventry
    {Vaster Inc.} % Organization
    {Web Application Developer} % Job title
    {Apr. 2017 - May. 2018} % Date(s)
    {North York, Ontario, CA} % Location
    {
      \begin{cvitems} % Description(s) of tasks/responsibilities
        \item {Learned and adapted Angular and \importantstyle{Symfony} (REST API) to develop Vaster Dashboard (VDP), was mainly designed to visualize Vaster Platform Data and support CRUD operations.}
          \begin{cvsubitems}
            \item {VDP was deployed on AWS \importantstyle{CentOS} EC2.}
          \end{cvsubitems}
        \item {Assisted in architecting and developing Cloud-based API of Novo Chat by learning and adapting Spring Boot and \importantstyle{Spring Cloud} and container technologies (\importantstyle{Docker}).}
          \begin{cvsubitems}
            \item {Implemented Discovery Service, Default Gateway, Configuration server and other microservices.}
            \item {Microservices were deployed on multiple AWS \importantstyle{Ubuntu} EC2 instances.}
          \end{cvsubitems}
        \item {Learned and adapted the best security practices and tools:}
          \begin{cvsubitems}
            \item {\importantstyle{JWT authentication} had been used to authenticate VDP users (access token and refresh token).}
            \item {Microservices were set to use JWT to talk to each other \importantstyle{over a secure channel}.}
          \end{cvsubitems}
      \end{cvitems}
      %\begin{cvsubentries}
      %  \cvsubentry{}{KNOX(Solution for Enterprise Mobile Security) Penetration Testing}{Sep. 2013}{}
      %  \cvsubentry{}{Smart TV Penetration Testing}{Mar. 2011 - Oct. 2011}{}
      %\end{cvsubentries}
    }

%---------------------------------------------------------
\end{cventries}

% \input{resume/honors.tex}
% \input{resume/presentation.tex}
% \input{resume/writing.tex}
% \input{resume/committees.tex}
% \input{resume/extracurricular.tex}
% %-------------------------------------------------------------------------------
%	SECTION TITLE
%-------------------------------------------------------------------------------
\cvsection{Skills \& Concepts}


%-------------------------------------------------------------------------------
%	CONTENT
%-------------------------------------------------------------------------------
\begin{cvskills}

%---------------------------------------------------------
\cvskill
  {Languages} % Category
  {\importantstyle{C}, C++, \importantstyle{Java}, \importantstyle{JavaScript}, \importantstyle{SQL}, \importantstyle{Python}, Typescript, Eiffel, \importantstyle{HTML}, CSS, PHP, smalltalk, drRacket, QBasic} % Skills

%---------------------------------------------------------
\cvskill
  {Concepts} % Category
  {\importantstyle{Linux}, Version-Control(\importantstyle{Git}), Concurrency, \importantstyle{Multithreading}, Synchronization, Microservices, REST APIs, Databases, \importantstyle{UX/UI}, VR, Game Engines (UE4, Unity)} % Skills

%---------------------------------------------------------
\cvskill
  {Frameworks} % Category
  {\importantstyle{Spring Boot}} % Skills

%---------------------------------------------------------
\end{cvskills}

% %-------------------------------------------------------------------------------
%	SECTION TITLE
%-------------------------------------------------------------------------------
\cvsection{Volunteer and Other Experience}


%-------------------------------------------------------------------------------
%	CONTENT
%-------------------------------------------------------------------------------
\begin{cvparagraph}

%---------------------------------------------------------
In junior school, began to learn \importantstyle{robotics} and led the Danesh High School team for three years. Team achieved the first place in two local tournaments and an award in \importantstyle{Iran Open Robo Cup 2012}. Taught high school-level robotics for almost a year after leading team to success in the Robo Cup before moving to Canada. While in high school and university, taught Java and PHP to friends, and built websites using \importantstyle{Drupal}, \importantstyle{WordPress} and \importantstyle{Joomla}.

Participated in the \importantstyle{Google Foobar} challenge and shared contact information with Google (\importantstyle{saaniaki@gmail.com}).
\end{cvparagraph}


%-------------------------------------------------------------------------------
\end{document}
